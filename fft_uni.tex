% Multidimensional FFT on Uniprocessors
%
\documentclass{article}
\usepackage{latexsym,amsmath,graphicx}
\textwidth 5in
%
\newif\ifshowcode
\showcodetrue

\title{The Inverse Shuffle Transpose and \\
       Multidimensional Fast Fourier Transforms}
\author{Alan H. Karp \\
        Hewlett-Packard Laboratories \\
        alan.karp@hp.com}

\begin{document}
\pagenumbering{roman}
\maketitle

\begin{abstract}

Combining the steps of a multi-pass transpose algorithm with the
butterflies of a Fast Fourier Transform (FFT) algorithm improves the
performance of the calculation of multi-dimensional FFTs on machines
with hierarchical memories.  This paper introduces the Inverse Shuffle
Transpose and shows how it can be combined with an extension of the
Pease FFT to produce a multidimensional FFT that avoids explicit
transposes yet accesses memory with good spatial locality.  This
approach has the additional advantages of being independent of the
number of dimensions and having uniform memory access.

\marginpar{Note from Kevin Wadleigh: discuss stride of access to
twiddle factors in Pease.}

\end{abstract}

\section{Introduction}
\pagenumbering{arabic}

Discrete Fourier transforms are important in a wide variety of
applications, from modeling underground formations in oil exploration
to the search for extraterrestrial intelligence.  Nevertheless, the
Fourier tranform would be not be the important computational tool it
is were it not for the Fast Fourier Transform (FFT)~\cite{cooley},
which reduces the complexity of a Fourier transform of length $N$ from
$O(N^2)$ to $O(N \log N)$ when $N$ is a power of 2.  The FFT relies
on the composite nature of $N$ and symmetries in the complex roots of
unity; the computational complexity is still $O(N^2)$ when $N$ is
prime.  (The Fractional FFT~\cite{fractional} has better scaling in
this case.)  A two dimensional FFT applies the algorithm separately to
each row and then to each column, with the obvious extension to more
dimensions.

The performance of algorithms on modern computers often depends more
on the memory access patterns than on the time to do the arithmetic.
Some machines have memory banks that can only deliver data every few
cycles; accessing a bank before it is ready results in a delay.
Others have set associative caches that have only a small number of
locations that can hold a particular piece of data~\cite{hennessey}.
FFT algorithms that involve power of two strides perform poorly on
such machines.  Other machines have hierarchical memories that deliver
blocks of data to successively smaller, but faster, cache storage
units.  FFT algorithms with little spatial or temporal locality in the
data references perform poorly on these machines.  Still other
machines have all these problems.

This paper is concerned with multidimensional FFTs on machines with
hierarchical memories.  Such machines perform well in either of two
modes, namely

\begin{itemize}
  \item algorithms that move data to cache and use it many times, and
  \item algorithms that stream the data through the processor, using
    every word tranferred at least once.
\end{itemize}
%
Algorithms with neither spatial nor temporal locality perform poorly.

There are one dimensional FFT algorithms, {\em e.g.}
Stockham~\cite{vanloan}, that access the data at small stride.
However, after computing the FFT of the rows at low stride, the next
step of the multidimensional calculation computes the FFT over the
columns, a large stride access.  Most implementations avoid this
problem by explicitly transposing the array before the second step,
and efficient transpose algorithms have been developed~\cite{vanloan}.
Others have developed block algorithms that do portions of the 2D FFT
while blocks of data are in cache.  Unfortunately, the code for such
implementations is complicated by the data blocking, and a different
version is needed for each number of dimensions.

Section~\ref{sec:transpose} introduces the {\em Inverse Shuffle
Transpose}.  This transpose is combined with a variant of the Pease
FFT in Section~\ref{sec:fft}.  In Section~\ref{sec:performance} the
performance of this approach is compared with several old and existing
implementations provided by computer vendors as part of their math
libraries.

\section{Inverse Shuffle Transpose}
\label{sec:transpose}

Transposing a 2D array is easy.  An NxM array can be transposed by

\begin{verbatim}
  for i = 1 to N
    for j = 1 to M
       B(i,j) = A(j,i)
\end{verbatim}
%
The problem with this approach is that array $A$ is accessed at large
stride.\footnote{Throughout this paper we'll assume that the arrays
are stored in row major order.}  Even worse, if the dimensions of $A$
are powers of two, as is common for FFTs, then we have added problems
with memory banks and cache access.  Some machines with hierarchical
memories allow bypassing the cache on stores, so it might be better to
access array $B$ at large stride.  However, even on these machines the
main memory often has banks.

Since the simplest approach performs poorly on many machines, we'll
need to do something more complicated.  One approach is to divide
arrays $A$ and $B$ into blocks that fit into the cache.  We can then
apply the simple algorithm for each block.  Using this approach also
allows us to to a transpose in place at the cost of a work array the
size of the block.

Although block transposes are quite efficient, Section~\ref{sec:fft}
shows that it is worth looking at multi-pass algorithms.  Here we'll
introduce the {\em inverse shuffle transpose}.  Recall that the
perfect shuffle of an array $X$ of even length $N$ is the mapping

\begin{verbatim}
  for i = 0 to N/2-1
    Y(2*i  ) = X(i)
    Y(2*i+1) = X(i+N/2)
\end{verbatim}
%
The name of this mapping comes from the analogy with cards -- cutting
the deck into equal parts and interleaving the resulting halves.  The
equivalent inverse shuffle is

\begin{verbatim}
  for i = 0 to N/2-1
    Y(i)     = X(2*i)
    Y(i+N/2) = X(2*i+1)
\end{verbatim}
%
which undoes the perfect shuffle.  The inverse shuffle is sometimes
referred to as the {\em even-odd sort permutation}.

\subsection{Radix 2 Transpose}
\label{sec:radix2}

If array $A$ has dimensions $N_1$ and $N_2$, where $N_1$ and $N_2$ are
powers of 2, we can think of $A$ as a linear array $X$ of $N = N_1
N_2$ elements.  The ($i_1$,$i_2$) element of $A$ is at location
$i_2+i_1N_2$ in $X$, where $0 \le i_1 < N_1$, and $0 \le i_2 < N_2$.
Letting $N_2 = 2^{n_2}$, we can transpose $A$ in $n_2$ inverse shuffle
steps.

The proof uses integer (truncating) arithmetic in which 7/2 = 3, and
the notation ``//'' for modulo arithmetic, {\em e.g.} 7//2 = 1.  Let
$k = i_2 + i_1N_2$, $0 \le k < N$, be the index of $A(i_1,i_2)$ in the
linear representation.  After $p<n_2$ inverse shuffle steps element
$k$ is at location

\begin{equation}\label{eqn:kp}
  k_p = k/2^p + (N/2^p) \sum_{r=0}^{p-1}2^r[(k/2^r)//2].
\end{equation}

The proof follows by induction.  The first inverse shuffle step, $p =
1$, moves element $k$ to location $k_1 = k/2$ if $k$ is even and to
$k_1 = k/2 + N/2$ if $k$ is odd.  We can represent this mapping
by

\begin{equation}
  k_1 = k/2 + (N/2)(k//2).
\end{equation}
%
Note that $k//2$ is the low order bit of $k$, which we'll denote
$d_0$.  Because $N_2$ is a multiple of 2, $k//2 = i_2//2$ making $d_0$
the low order bit of $i_2$ as well.  Equation~\ref{eqn:kp} also
produces this result.

The second step, $p = 2$, moves element $k_1$ to position $k_2 =
k_1/2$ if $k_1$ is even and to $k_1/2 + N/2$ if $k_1$ is odd.  As
before, we have

\begin{equation}
  k_2 = k/4 + (N/4) d_0 + (N/2)[(k/2)//2].
\end{equation}
%
We note that $d_1 = (k/2)//2 = (i_2/2)//2$ is the second least
significant bit of $i_2$ because $N_2$ is a multiple of 4.  Again, we
see that Equation~\ref{eqn:kp} satisfies this equation.

For step $p+1$, we have

\begin{equation}
  k_{p+1} = k/2^{p+1} +
            (N/2^{p+1})\sum_{r=0}^{p-1} d_r 2^r + 
            (N/2)[(i_2/2^p)//2],
\end{equation}
%
which is

\begin{equation}
  k_{p+1} = k/2^{p+1} + (N/2^{p+1})\sum_{r=0}^{p} d_r 2^r,
\end{equation}
%
completing the proof.

Applying this equation to find the location of element $k$ after $n_2$
steps we see that

\begin{equation}
  k_{n_2} = k/2^{n_2} + N_1(N_2/2^{n_2}) \sum_{r=0}^{n_2-1} d_r 2^r.
\end{equation}
%
Using the facts that $N_2 = 2^{n_2}$, $k = i_2 + i_1N_2$ and $i_2 <
N_2$ gives

\begin{equation} \label{eqn:radix2}
  k_{n_2} = i_1 + N_1 \sum_{r=0}^{n_2-1} d_r 2^r.
\end{equation}
%
We see that $k_{n_2} = i_1+i_2N_1$ because the summation in
Equation~\ref{eqn:radix2} is just the binary representation of $i_2$.
Hence, $k_{n_2}$ is at the location of $A(i_2,i_1)$ when that array is
treated as stored in row major order.

\subsection{Mixed Radix}
\label{sec:mixedradix}

The inverse shuffle generalizes to mixed radix problems.  For a factor
$f$, we take $f$ elements and move them to every $N/f$th position.  In
other words, element $(i_1,i_2)$, which is at location $k =
i_2+i_1N_2$ in the linear representation, goes to location $k_1 = k/f
+ (N/f)(k//f)$.

Say that $N_2 = \prod_{j=1}^F f_j$, where the factors $f_j$ need not
be prime.  Then for $1 \le p \le F$, the $p$'th inverse shuffle step
is

\begin{equation}\label{eqn:mixedradix}
  k_p = k/F_p + N/F_p \sum_{r=0}^{p-1} d_r F_r,
\end{equation}
%
where $F_p = \prod_{j=1}^p f_j$, $F_0 = 1$, and $0 \le d_r < f_r$ is
the $r$th digit of both $k$ and $i_2$.  Hence, $k_F = i_1 + i_2N_1$,
as in the radix-2 case.  When $f_1 = N_2$, the inverse shuffle is
exactly the simple transpose presented in Section~\ref{sec:transpose}.

\subsection{Higher Dimensionality}

An important property of the inverse shuffle transpose is that it
affects only the last dimension of the mapping.  That means that the
first dimension in the representation can be a composite of all the
remaining dimensions of a multidimensional array.  Hence, if the array
has dimensions $N_q$, $1 \le q \le Q$, and we apply the inverse
shuffle transpose, the result is a cyclic permutation of the
dimensions.  In other words, we end up with an array of dimension
$N_Q, N_1, N_2, \ldots, N_{Q-1}$.  Repeating the process $Q$ times
returns the data to its original order.

We can see this behavior by applying the inverse shuffle to an array
with $Q$ dimensions.  Element $A(i_1, \ldots, i_Q)$ is at location

\begin{equation}
  k = N_Q \sum_{j=1}^{Q-2} i_j \prod_{m=j+1}^{Q-1} N_m + N_Q i_{Q-1} + i_Q
\end{equation}
%
in the linear representation.  For example, a three dimensional array
element at $A(i_1,i_2,i_3)$ is at location $k = i_1N_3N_2 + i_2N_3 +
i_3$ in the linear representation.

If $N_Q = \prod_{j=1}^F f_j$, the $F$'th inverse shuffle step moves
element $k$ to position

\begin{equation}\label{eqn:multidim}
  k_F = \sum_{j=1}^{Q-2} i_j \prod_{m=j+1}^{Q-1}N_m + i_{Q-1} +
        (N/N_Q) \sum_{r=0}^{p-1} d_r F_r .
\end{equation}
%
The second summation is the decomposition of $i_Q$, so we see that
this index has moved to the first position in the representation, and
al the others are shifted one position.

\section{Multidimensional FFT}
\label{sec:fft}

There are many ways to compute the FFT.  Some can be done without
requiring additional storage; some require a bit reversal; others
operate at low stride.  Here we'll use a variant due to Pease that
has the same memory access pattern as the inverse shuffle transpose.

\subsection{Radix-2 Pease Formulation} \label{sec:pease2}

The Pease formulation of the FFT of a 1D array $x$ of length $N = 2^n$
starts with a bit reversal permutation, denoted by the permutation
$P_N$.  It then performs $n$ {\em butterfly} stages.

\begin{verbatim}
  for f = 1 to n
    for j = 0 to N-1
      if f == 1
        y(j) = x(P(j))
      else
        y(j) = x(j)
    for j = 0 to N/2-1
      x(j)     = y(2*j) + W(f,j)*y(2*j+1)
      x(j+N/2) = y(2*j) - W(f,j)*y(2*j+1)
\end{verbatim}
%
where $W(f,j)$, the {\em twiddle factors}, are defined in
Section~\ref{sec:twiddle2}.  At the end, the result is stored in array
$x$.  This pseudo-code is not the most efficient implementation.  In
particular, the first loop over $j$ is not needed.  The exchance can
be done by pointer swapping, and the bit reversal can be combined with
the first butterfly.

There are two disadvantages to the Pease formulation.  First, it can
not be done in place, which limits the size of arrays that can be
tranformed using an in-memory algorithm.  This problem is somewhat
mitigated by the fact that only enough extra space is needed for half
the array because the first half can be reused.

The second disadvantage of the Pease formulation is that it requires a
bit reversal.  However, bit reversals for multidimensional FFTs should
not be a problem.  Consider a 3D FFT that fills the 1 GB memory of a
machine.  If the dimensions are of equal size, then each bit reversal
involves only 1 KB of data, an amount small enough to fit in cache.
Even a 2D FFT of this total size has rows of 32 KB, which also fit in
cache.  Efficient bit reversals can be done, even when the row doesn't
fit in cache~\cite{bitrev}.

\subsection{Radix-2 Twiddle Factors} \label{sec:twiddle2}

The array of twiddle factors, $W$, is a particular arrangement of the
roots of unity, $\omega_N^j = \exp(2 \pi i j/N)$, where $i =
\sqrt{-1}$.  The first thing to note is that each row of $W$ has
length $N/2$ because $W(f,j+N/2) = - W(f,j)$ for $0 \le j < N/2$.
With this observation, a convenient representation is to write $W(f,j)
= (\omega_N^j)^{N/2^f}$, with $1 \le f \le n$.  Hence, the first row
of $W$ consists of all ones.  The first half of the second row is ones
and all the elements in the second half are $i$, the positive fourth
root of unity.  The last row of the array consists of the roots of
unity with positive imaginary part.

Although $W$ doesn't consume too much space, we can access the same
numeric values by properly computing the index into a list of the
roots of unity.  In particular, replacing $W(f,j)$ in the pseudo-code
with $w(j/(N/2^f))$, where $w(j) = \omega_N^j$, gives us the desired
coefficients.  We need not worry about the strided access to $w$
because the array will fit in cache for the multidimensional problems
we're considering.

\subsection{Radix-2 Multidimensional FFT} \label{sec:mdfft2}

The data access pattern in the radix-2 Pease formulation is nearly
identical to that of the inverse shuffle transpose.  The only change
is to store the odd numbered terms, those computed from the
difference, in the second half of the output array instead of the
second half of the row being computed.

\begin{verbatim}
      for q = 1 to Q
         w = roots(n(q))
         for f = 1 to n(q)
            do j = 0, N - 1
               if q == 1
                  k = mod(j,N(q))
                  y(j) = x(P(k))
               else
                  y(j) = x(j)
            do j = 0 to N/2 - 1
               k = (N(q)/2**f)*(j/(N/2**f))
               x(j    ) = y(2*j)+w(k)*y(2*j+1)
               x(j+N/2) = y(2*j)-w(k)*y(2*j+1)
\end{verbatim}
%
There are only a few differences from this code and that in
Section~\ref{sec:pease2}.  First, there is a loop over the number of
dimensions.  Also, the bit reversals are done separately for each row
by computing the index modulo the length of a row.  Finally, the
indexing of the twiddle factor array is more complicated.

This last point warrants discussion.  In Section~\ref{sec:twiddle2} we
used an index of $j/(N/2^f)$, where $N$ in that case corresponds to
$N_q$ in the multidimensional case.  In the pseudo-code, though, we
use an index of $(N_q/2^f)(j/(N/2^f)$ because the elements move
relative to the array as a whole instead of relative to the row.

\subsection{Mixed Radix Pease} \label{sec:pease}

It shouldn't come as a surprise to find out that the data access
pattern for the mixed radix Pease forumulation is the same as that of
the mixed radix inverse shuffle transpose.  There are two issues that
we need to deal with, the generalization of the bit reversal and the
handling of the twiddle factors.  In both cases there is an ambiguity
in what factorization is being used.  For example, 96 can be factored
as $(32,3), (3,32), (8, 4, 3), (3, 8, 4), \ldots$.  Rather than pick a
factorization at random, we require the programmer to specify each
dimension as an array of factors to be used.  In terms of the notation
used earlier, we write $N_q = \prod_{f=1}^F N_{qf}$.

We need to implement a digit reversal given a factorization of $N_q$.
Since we're interested in bit reversing sequences that fit in cache,
we'll take the direct approach of computing an index vector that can
be used for a gather operation.  We also take a brute force approach
to computing this index vector.  The $j$th element in the row is at

\begin{equation}
  j = j_1 + \sum_{f=2}^F j_f \prod_{k=2}^f N_{qk} .
\end{equation}
%
After reordering that element will be at position

\begin{equation}
  j_r = j_F +\sum_{f=1}^{F-1} j_{F-f} \prod_{k=F-f}^{F-1} N_{qk}
\end{equation}
%
in the digit reversed array.  The permutation becomes $P_{N_q}(j) =
j_r$.

We can use a very simple modification of the way we get the twiddle
factors for radix-2 problems.  Instead of using a factor of $2^f$ we
use $M_{qf} = \prod_{k=1}^f N_{qk}$ when computing an index at the
$f$th butterfly.  Hence, the twiddle factor we use at butterfly $f$ is
$w((N_q/M_{qf})(j/(N_q/M_{qf})))$.

\section{Performance}
\label{sec:performance}

\bibliographystyle{plain}

\begin{thebibliography}{99}

\end{thebibliography}

\appendix

%\begin{figure}[htb]
%  \fbox{
%  \includegraphics[width=4.75in,keepaspectratio=true] 
%                  {engagement_dealspace.pdf}}
%  \caption{Effect of moving through the deal space.}
%  \label{fig:dealspace}
%\end{figure}

\end{document}