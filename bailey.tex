\documentstyle[12pt,fleqn]{article}
\setlength{\oddsidemargin}{0cm}
\setlength{\textwidth}{16.2cm}
\setlength{\columnwidth}{16.2cm}
\setlength{\topmargin}{0cm}
\setlength{\textheight}{21.5cm}

\begin{document}

\vspace*{2.7cm}
\begin{large} \begin{center}
A Fast Method for the Numerical Evaluation of Fourier Transforms \\ 
David H. Bailey and Paul N. Swarztrauber \\ 
RNR Technical Report RNR-XX-XXX \\
June 30, 1992
\end{center} \end{large}

\vspace{3ex} \noindent
{\bf Abstract}

The fast Fourier transform (FFT) is often used to compute numerical
approximations to the continuous Fourier transform.  However, a
straightforward application of the FFT to this problem typically
requires a large FFT to be performed, even though most of the input
data to this FFT are zero and only a small fraction of the output data
are of interest.  In this note, the ``fractional Fourier transform'',
previously developed by the authors, is applied to this problem with a
substantial savings in computation.

\vfill{
Bailey is with the Numerical Aerodynamic Simulation (NAS) Systems
Division at NASA Ames Research Center, Moffett Field, CA 94035.
Swarztrauber is with the National Center for Atmospheric Research,
Boulder, CO 80307, which is sponsored by National Science Foundation.}

\newpage
\noindent {\bf 1. Introduction}

The continuous Fourier transform (CFT) of a function $f(t)$ (which may
have real or complex values) and its inverse will be defined here as
\begin{eqnarray}
F[f](x) &=& \int _{-\infty} ^\infty f(t) e^{-2 \pi i t x} dt \\
F^{-1}[f](x) &=& \int _{-\infty} ^\infty f(t) e^{2 \pi i t x} dt
\end{eqnarray}

\noindent
where $i$ is the imaginary unit.  The discrete Fourier transform (DFT)
and the inverse DFT of an $n$-long sequence $z$ (which may have real
or complex values) will be defined here as
\begin{eqnarray}
D_k (z) &=& \sum _{j = 0} ^{n - 1} z_j e^{-2 \pi i j k / n} \\
D_k^{-1} (z) &=& \frac{1}{n} \sum _{j = 0} ^{n - 1} z_j e^{2 \pi i j k / n}
\end{eqnarray}

Straightforward evaluation of the DFT using these formulas is expensive,
even when the exponential factors are precomputed.  This cost can be
greatly reduced by employing one of the variants of the fast Fourier
transform (FFT) algorithm \cite{dhb-fft1,dhb-fft2,swarz-fft,vanloan}.

\vspace{2ex}
\noindent {\bf 2. A Conventional Scheme for Evaluating CFTs with FFTs}

There are of course a number of advanced techniques for the
approximate numerical evaluation of definite integrals \cite{davis}.
Unfortunately, most of these techniques deal with evaluating only one
integral, rather than a large number of integrals, each with a
different integrand.  Evaluating the CFT falls in this latter
category, since one usually requires the values of this integral for
large range of $x$.

The FFT can be effectively applied to this problem as follows.  Let us
assume that the function $f(t)$ is zero outside of an interval $[-b/2,
b/2]$, an assumption that is often made in practical applications.
For a large positive integer $n$ divisible by four, define $h = b /
n$, and for integers $j$ and $k$, let $t_j = (j - n / 2) h$ and $x_k =
(k - n / 2)/ b$.  Then we can write
\begin{eqnarray}
F(x_k) &=& \int _{-\infty} ^\infty f(t) e^{-2 \pi i t x_k} dt \\
     &=& \int _{-b/2} ^{b/2} f(t) e^{-2 \pi i t x_k} dt \\
     &\approx& \sum _{j = 0} ^{n - 1} f(t_j) e^{-2 \pi i t_j x_k} h \\
     &=& h \sum _{j = 0} ^{n - 1} f(t_j) e^{-2 \pi i (j - n / 2) 
       (k - n / 2) / n} \\
     &=& \frac{b}{n} e^{\pi i (k - n / 2)} \sum _{j = 0} ^{n - 1} 
       f(t_j) e^{\pi i j} e^{-2 \pi i j k / n} \\
     &=& (-1)^k \frac{b}{n} D_k [\{ (-1)^j f(t_j) \}] \hspace{20mm} 
       0 \leq k < n
\end{eqnarray}

The DFT indicated in (10) can of course be rapidly evaluated using an
FFT.  If one wishes to obtain accurate, high-resolution results from
this computation, then $b$ must be large, and $n$ must be very large.
To be precise, note that results are produced at an interval in $x$ of
$1 / b$.  Presuming that a comparable interval is required in $t$ in
order to obtain accurate results, this means that $h \approx 1 / b$,
or in other words that $n \approx b^2$.

This scheme is reasonable straightforward and relatively efficient.
However, one is struck by the fact that to meet a given accuracy and
resolution requirement, $b$ may have to be set substantially greater
than the length the interval where $f(t)$ is nonzero.  Thus
potentially a large fraction of the input data to the FFT used to
evaluate (15) may be zero.  Also, only a small fraction (the central
values) of the output of this FFT may be of interest.

\vspace{2ex}
\noindent {\bf 3. The Fractional Fourier Transform}

We will employ here a generalization of the DFT that has been termed
the fractional Fourier transform (FRFT) \cite{frft}.  It is defined on
the $m$-long complex sequence $x$ as
\begin{eqnarray}
G_k (x, \alpha) &=& \sum_{j=0}^{m-1} x_j e^{-2 \pi i j k \alpha}
\end{eqnarray}

\noindent
The parameter $\alpha$ is not restricted to rational numbers and in
fact may be any complex number.  Note that the ordinary DFT and its
inverse are special cases of the fractional Fourier transform.

A fortunate fact is that the FRFT admits computation by means of a
fast algorithm, one with complexity comparable to the FFT.  In fact, it
employs the FFT in a crucial step.  This algorithm can be stated as
follows.  Define the $2 m$-long sequences $y$ and $z$ as
\begin{eqnarray}
y_j &=& x_j e^{-\pi i j^2 \alpha} \hspace{23mm} 0 \leq j < m \\
y_j &=& 0 \hspace{36mm} m \leq j < 2 m \\
z_j &=& e^{\pi i j^2 \alpha} \hspace{29mm} 0 \leq j < m \\
z_j &=& e^{\pi i (j - 2 m)^2 \alpha} \hspace{20mm} m \leq j < 2 m
\end{eqnarray}

\noindent
It is then shown in \cite{frft} that
\begin{eqnarray}
G_k (x, \alpha) &=& e^{- \pi i k^2 \alpha} E_k^{-1} [\{ E_j (y) 
     E_j (z) \}] 
   \hspace{2cm} 0 \leq k < m
\end{eqnarray}

\noindent
The remaining $m$ results of the final inverse DFT are discarded.
Element-wise complex multiplication is implied in the expression $E_j
(y) E_j (z)$.  These three aliased DFTs can of course be efficiently
computed using $2 m$-point FFTs.

To compute a different $m$-long segment $G_{k+s} (x, \alpha), \; 0
\leq k < m$, it is necessary to slightly modify the above
procedure.  In this case $z$ is as follows:
\begin{eqnarray}
z_j &=& e^{\pi i (j + s)^2 \alpha}  \hspace{27mm} 0 \leq j < m \\
z_j &=& e^{\pi i (j + s - 2 m)^2 \alpha} \hspace{20mm}
    m \leq j < 2 m
\end{eqnarray}

\noindent
Then we have
\begin{eqnarray}
G_{k+s} &=& e^{- \pi i (k + s)^2 \alpha} E_k^{-1} [\{ E_j (y) E_j (z) \}] 
   \hspace{20mm} 0 \leq k < m
\end{eqnarray}

Note that the exponential factors in (12) through (19) can be
precomputed.  Furthermore, the DFT of the $z$ sequence can also be
precomputed.  Thus the cost of an $m$-point FRFT is only about four
times the cost of an $m$-point FFT.

\vspace{2ex}
\noindent {\bf 4. Computing CFTs with FRFTs}

One of the applications of FRFTs mentioned in \cite{frft} is computing
DFTs of sparse sequences, i.e. sequences that are mostly zero.  It is
particularly effective when only a small portion of the input data
values are nonzero, and only a small portion of the output values are
required.  Thus, the FRFT is very well suited for the problem at hand.

Let us now suppose that $f(t)$ is nonzero for $[-a / 2, a / 2]$, where
$a$ is smaller than the $b$ used in the previous section.  In
particular, let us assume that $b = q a$ and $n = q m$ for integers
$q$ and $m$.  We now set $h = b / n$ as before, but define $t_j = (j -
m / 2) h$ and $x_k = (k - m / 2)/ b$ for integers $j$ and $k$.  Then
we can write
\begin{eqnarray}
F(x_k) &=& \int _{-\infty} ^\infty f(t) e^{-2 \pi i t x_k} dt \\
     &=& \int _{-a/2} ^{a/2} f(t) e^{-2 \pi i t x_k} dt \\
     &\approx& \sum _{j = 0} ^{m - 1} f(t_j) e^{-2 \pi i t_j x_k} h \\
     &=& h \sum _{j = 0} ^{m - 1} f(t_j) e^{-2 \pi i (j - m / 2) 
       (k - m / 2) / n} \\
     &=& \frac{b}{n} e^{\pi i (k - m / 2) m / n} \sum _{j = 0} ^{m - 1} 
       f(t_j) e^{\pi i j m / n} e^{-2 \pi i j k / n} \\
     &=& \frac{b}{n} e^{\pi i (k - m / 2) / q} G_k [\{f(t_j) 
       e^{\pi i j / q} \}, 1 / n] \hspace{20mm} 0 \leq k < m
\end{eqnarray}

We now have an economical means of computing the central $m$ results
of this transform.  Additional $m$-long segments of results can be
obtained by applying the more general from of the FRFT given at the
end of the previous section.

\vspace{2ex}
\noindent {\bf 5. Implementation Example}

Collecting the results we have obtained so far, we have
\begin{eqnarray}
F(x_k) &=& \int _{-\infty} ^\infty f(t) e^{-2 \pi i t x_k} dt \\
       &\approx& \frac{b}{n} e^{\pi i (k - m / 2) / q} 
       G_k [\{f(t_j) e^{\pi i j / q} \}, 1 / n] \\
       &=& \frac{b}{n} e^{\pi i [(k - m / 2) / q - k^2 / n]}
         D_k^{-1} [\{ D_j (y) D_j (z) \}] \hspace{20mm} 0 \leq k < m
\end{eqnarray}

\noindent
where the $2m$-long sequences $y$ and $z$ are given by
\begin{eqnarray}
y_j &=& f(t_j) e^{\pi i (j / q - j^2 / n)} \hspace{20mm} 0 \leq j < m \\
y_j &=& 0 \hspace{47mm} m \leq j < 2 m \\
z_j &=& e^{\pi i j^2 / n}  \hspace{38mm} 0 \leq j < m \\
z_j &=& e^{\pi i (j - 2m)^2 / n} \hspace{29mm}
    m \leq j < 2 m
\end{eqnarray}

Let us now consider the problem of numerically computing the Fourier
transform of the Gaussian probability density function
\begin{eqnarray}
f(t) &=& \frac{1}{\sqrt{2 \pi}} e^{-t^2/2}
\end{eqnarray}

\noindent
It is well known that the Fourier transform of this function is
\begin{eqnarray}
F(t) &=& e^{-2 \pi^2 x^2}
\end{eqnarray}

\noindent
Outside the interval $[-8, 8]$, the function $f(t) < 5.1 \times
10^{-15}$ and thus can be considered to be zero to high accuracy.
Suppose that we desire a resolution of $1 / 256$ in both the $t$ and
$x$ variables.  Then in terms of the notation we have been using, we
have $a = 16, \; b = 256, \; n = 65536, \; m = 4096$, and $q = 16$.

The computation defined in (26) through (32) was completed on a
Silicon Graphics Personal Iris workstation in 0.89 seconds, as
compared with 4.43 seconds using the conventional FFT scheme given in
(5) through (10).  Comparing the results of the FRFT scheme with the
exact formula given in (34), the root-mean-square (RMS) error is $4.47
\times 10^{-16}$.  The RMS error of the central $m$ results obtained
with conventional FFT scheme is $4.67 \times 10^{-17}$.  Both of these
errors are virtually at the limit of machine precision for 64-bit
data.

\begin{thebibliography}{5}

\bibitem{dhb-fft1} D. H. Bailey, ``FFTs in External or Hierarchical
Memory'', {\sl Journal of Supercomputing}, vol. 4 (1990), p. 23 - 35.

\bibitem{dhb-fft2} D. H. Bailey, ``A High-Performance FFT Algorithm for
Vector Supercomputers'', {\sl International Journal of Supercomputer
Applications} vol. 2 (1988), p. 82 - 87.

\bibitem{frft} D. H. Bailey and P. N. Swarztrauber, ``The Fractional
Fourier Transform and Applications'', {\sl SIAM Review}, vol. 33, no.
3 (Sept.  1991), p. 389 - 404.

\bibitem{davis} P. Davis and P. Rabinowitz, ``Methods of Numerical
Integration'', Wiley, New York, 1975.

\bibitem{swarz-fft} P. N. Swarztrauber, ``Multiprocessor FFTs'', {\sl
Parallel Computing}, vol. 5 (1987), p. 197 - 210.

\bibitem{vanloan} C. Van Loan, {\sl Computational Frameworks for the
Fast Fourier Transform}, SIAM, Philadelphia, 1992.

\end{thebibliography}

\end{document}


